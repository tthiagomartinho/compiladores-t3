\documentclass[
12pt,				% Tamanho da fonte.
a4paper,			% Tamanho do papel.
english,			% Idioma adicional para hifenização.
french,				% Idioma adicional para hifenização.
spanish,			% Idioma adicional para hifenização.
brazil,				% O último idioma é o principal do documento.
article
]{abntex2}



%%% Importação de pacotes. %%%

% Codificação do documento em Unicode.
\usepackage[utf8]{inputenc}

\usepackage[T1]{fontenc}

%opening
\title{Trabalho 3 Compiladores}
\author{Rodrigo Pimenta, Thiago Martinho}

\begin{document}

\maketitle

\section{Introdução}

Este trabalho teve como objetivo a construção do analisador semântico do  compilador proposto para a linguagem GPortugol cujo manual pode ser encontrado em (\url{http://www.inf.ufes.br/~mberger/Disciplinas/2015_2/Compiladores/manualGPortugol.pdf}). A base desse trabalho de número três é o analisador léxico e sintático construídos com o auxílio das ferramentas Flex e Bison respectivamente.
Para o desenvolvimento do analisador semântico, foi utilizado a linguagem de programação C e a ferramenta Bison.


\section{Analisador Semântico}

A construção de um analisador semântico é a terceira etapa do processo de codificação de um compilador para uma linguagem qualquer. Este analisador tem por objetivo verificar os erros semânticos contidos no código a ser compilado. 

Nesta etapa de compilação são verificados erros de tipo, de utilização de variáveis não declaradas, de chamadas e declarações de funções, de expressões, entre outros. Para os casos onde não há erro semântico, as variáveis declaradas já podem ser alocadas na memória a fim de diminuir o tempo de execução do programa.  

\section{Tabela de Variáveis e Funções}

Para realizar o controle de variáveis e funções foram criadas duas tabelas hash com um tempo de acesso constante de tamanho fixo de 97 posições. Para cada elemento declarado no programa, foi criado um registro na tabela posicionado de acordo com a função hash utilizada que calcula o resto da divisão entre a soma do equivalente ASCII de cada letra do nome do elemento e o tamanho da tabela. As colisões foram tratadas utilizando lista simplesmente encadeada, sendo que um novo elemento é inserido no começo da lista.

A tabela de variáveis apresenta a seguinte estrutura:

\begin{itemize}
	\item nome;
	\item valor;
	\item escopo;
	\item tipo;
	\item se a variável é usada ou não;
	\item dimensão da matriz;
	\item tamanho máximo de elementos para cada dimensão da matriz; 
\end{itemize}

Para os elementos que não são matrizes, os campos que representam as dimensões não são preenchidos;

Já a tabela de funções contém:

\begin{itemize}
	\item nome;
	\item tipo de parâmetros;
	\item tipo de retorno;
	\item aridade;
\end{itemize}

Para simplificação e padronização do programa, foram definidas as seguintes constantes que identificam os tipos primitivos:

\begin{table}[h]
	\caption{Constantes que identificam os tipos de variáveis primitivos}
	\label{tbl-divisao-sistema}
	\centering
	\begin{tabular}{| p{47mm} | p{100mm} | }\hline		
		\textbf{Tipo} & 
		\textbf{Valor}\\
		\hline
		TIPO VOID & -1\\
		\hline
		TIPO REAL & 0\\
		\hline
		TIPO INTEIRO & 1\\
		\hline
		TIPO CARACTERE & 2\\
		\hline
		TIPO LITERAL & 3\\
		\hline
		TIPO LOGICO & 4\\
		\hline
	\end{tabular}
\end{table}

\section{Erros semânticos}
Nesta seção, estão listados e descritos todos os possíveis erros semânticos na escrita de um programa em GPortugol.

\subsection{Variável redeclarada}
Uma mesma variável não pode ser redeclarada num mesmo escopo, logo quando isso ocorrer nosso programa informará qual a linha que uma variável foi redeclarada.

\subsection{Expressão}
Uma expressão é composta por operadores aritméticos ou operadores lógicos, não sendo possível uma mesma expressão conter os dois tipos. Caso, na expressão, exista elementos de tipos diferentes, uma mensagem de erro é exibida e o programa abortado.

\subsection{Matrizes}
Caso o usuário tente acessar uma posição de uma variável que não seja do tipo matriz, uma mensagem de erro é exibida e o programa abortado. A mesma coisa ocorre, caso uma posição inválida da matriz tente se acessada.

\subsection{Operadores incremento e decremento}
Apenas variáveis do tipo inteiro ou real podem estar associadas aos comandos ++ ou \textit{$-$$-$}.

\subsection{Função redeclarada}
Duas funções não podem ter o mesmo nome, se este caso existir nosso programa tratará isso como erro e informará a linha de tal redeclaração.

\subsection{Retorno de função}
Para os retornos de função ou elas retornam o tipo \textit{void} ou elas retornam um outro tipo qualquer. Uma função do tipo \textit{inteiro} só poderá ter como retorno um variável desse tipo, caso contrário será identificado um erro no comando retorno. Da mesma forma, se uma função não possuir um tipo e tiver como retorno uma variável do tipo \textit{inteiro} nosso programa informará um erro.

\subsection{Chamada de função}
Quando declaramos uma função informamos qual o tipo de retorno dela, qual seu nome e quais são os parâmetros que estamos esperando. Quando chamamos uma função devemos prestar atenção em duas coisas, a primeira delas é se a função existe, ou seja, não podemos chamar a função \textit{dobrar} se a mesma não tiver sido declarada anteriormente no programa. Em segundo lugar se a função espera dois parâmetros e passarmos somente um, ou se a função espera valores do tipo \textit{inteiro} e passarmos valores do tipo \textit{logico} nossa chamada de função estará incorreta. Nosso programa deverá identificar esses dois erros e informar ao usuário a linha na qual esse erros ocorreram, especificando qual tipo de erro ocorreu (" Função não declarada " ou " parâmetros passados na chamada de função não condizem com a declaração ").

\subsection{Atribuição de tipos diferentes}
Uma variável do tipo \textit{inteiro} não pode receber o valor \textit{verdadeiro}, por exemplo, da mesma forma que uma variável do tipo \textit{logico} não pode receber um número ou algo do tipo. Logo quando houver esse tipo de erro nosso programa o identificará e mostrará a linha que esse erro ocorre.

\subsection{Comando \textit{Para}}
O comando \textit{para} possui a restrição de que a variável auxiliar de iteração deve ser do tipo \textit{inteiro}, logo qualquer tipo diferente disso devemos interromper o programa e informar um erro.

\subsection{Comando \textit{Avalie}}
O comando \textit{Avalie} deve ser utilizado somente para variáveis do tipo \textit{inteiro}, logo qualquer variável que não seja desse tipo devemos informar um erro.

\section{Considerações em relação a gramática da linguagem}
Em relação ao analisador sintático desenvolvido no trabalho passado, a gramática do nosso compilador teve algumas alterações, sendo importante destacar:

\begin{itemize}
	\item Não é possível fazer várias chamadas de funções em cadeia. Exemplo: \textit{imprima(fatorial(leia())))};
	\item Não é possível ter chamadas de funções contendo expressões. Exemplo: \textit{fatorial(x-1)};
	\item O comando \textit{faca - enquanto} não mais requer parenteses na expressão a ser avaliada. O item passa a ser opcional;
	\item O comando imprima pode ter apenas um parâmetro;
	\item Não é possível acessar uma matriz utilizando identificadores. Apenas números inteiros podem ser utilizados como índices;
	\item Comando \textit{leia()} tem que ser utilizado junto com o comando atribuição;
	\item No comando \textit{AVALIE - CASO} só números inteiros podem ser utilizados como índices;
\end{itemize}


\end{document}
